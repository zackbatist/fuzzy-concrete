This paper draws from observations of and interviews with archaeologists at work, as well as the documents that they produced.
Specifically, I articulate how archaeologists enacted various activities and how their actions were situated as part of broader systems of knowledge production.
My involvement with this project constituted a longitudinal investigation of archaeological practice that contributed to my doctoral dissertation \parencite[]{batist2023a}.

\subsection*{Case}
I base my findings on a singular case -- that of a research project comprising excavation of a prehistoric site in Southern Europe.
It is directed by a foreign professor affiliated with a North American university, who coordinates various specialists whom the director recruited for their expertise in the interpretation of finds, a number of trench supervisors who lead excavation and coordinate data collection, and excavators who are usually less experienced students who operate under the guidance of their assigned trench supervisors.

I actively contributed to the project for several years, primarily serving as a database manager.
I documented how participants engaged with this project's information system from 2017 to 2019, which involved recording and interviewing archaeologists as they worked during the summer field seasons and holding additional interviews between fieldwork sessions.
The project's director also provided access to all documents and records for the purpose of this research.

This project served as a useful case study that illustrated the pragmatic and multifaceted ways in which participants reasoned and worked their way through the rather mundane activities that archaeologists commonly undertake in similar research contexts.
In case-study research, cases represent discrete instances of a phenomenon relating to researcher's interest \parencite[]{ragin1992}.
Cases are therefore not the subjects of inquiry, but the vehicles through which phenomena of interest are manifested in an observable way.
I recognize that all archaeological projects are informed by their own histories, memberships, sets of tools, methods, and social or political circumstances, which inform distinct traditions of practice, and that it is not possible to generalize across the whole discipline through a single case study.
However, I am able to articulate some significant factors that contribute to decisions and behaviours that archaeologists commonly make and enact, and to make certain underappreciated social and collaborative commitments that underlie common tools and practices more visible.
I am therefore able to draw attention to certain patterns of practice that relate to contemporary discourse on the nature of archaeological data and ongoing development of information infrastructures.

As such, my conclusions are informed by the informants whose actions and attitudes I sought to articulate, and by my own perspective as a scholar of the culture and practice of archaeology and of the media and infrastructures that support it.
One implication is that commercial archaeology, which comprises the vast majority of archaeological work in North America and Europe, is out of the study's scope, owing to the fact that the case represents a research project and that I have very limited experience with and knowledge about commercial archaeology.
However, see \textcites[]{chadwick1998}[]{thorpe2012}[]{zorzin2015} for similar research pertaining to commercial archaeology.

\subsection*{Data}
My dataset comprises recorded observations, embedded interviews, retrospective interviews, archaeological documentation, and ethnographic and reflexive fieldnotes.

Observational data comprised records of participants' behaviours as they performed various archaeological activities and take the form of video, audio and textual files.
They enable me to document \emph{how} practices are performed, in addition to the fact \emph{that} they are performed.
Moreover, observational data allow me to document what participants actually do as opposed to what they think or say they do.
For instance, I situated activities in relation to broader systems even when participants are unaware that they are contributing to these systems, and to consider how activities occurring at various times or in various contexts indirectly relate to, compare with or inform each other.
Some of the primary foci of my observations were the processes that result in archaeological records; people's use of information objects or interfaces, which sometimes differ from expected behaviour established through their design; how subjects implemented unconventional solutions or ``hacks'' to work around problems; how the context of an activity affects its implementation; and how local or idiosyncratic terms, concepts and gestures become established in a research community.

Embedded interviews comprised conversational inquiries with participants in the context of their work, and were meant to account for participants' perspectives regarding how and why they act as they do, given the immediate constraints of the situation at hand.
Embedded interviews provided insight into the practicalities of work in the moment, from the perspective of practitioners themselves \parencites[]{flick1997}[]{flick2000}[]{witzel2000}.
They are also useful for comparing participants' responses with observational records to interrogate how and why participants' observed actions may differ from the rationales elucidated from embedded interviews.
Some of the primary foci of my embedded interviews are to account for how participants identify problems or challenges in their work, and to determine ways to resolve them; how certain people gain recognition as domain experts or authorities with specialized knowledge; how specialists relate their contributions to the contributions of others; and how specialists relate their situated perspectives to centralized knowledge repositories.

Retrospective interviews comprised longer interviews outside of work settings with select participants to contextualize data collected by other means and to determine participants' views on more general or relatively unobservable aspects of archaeological research (such as planning, publishing, collaboration, etc).
They helped me gain insight into how participants situate themselves as members of and in relation to research communities, which may be characterized by different regimes of value and by different methodological protocols or argumentation strategies.
Some of the primary foci of my retrospective interviews are to highlight participants' perspectives on the value of various kinds of research outputs, what they value in their work and the work of others, the major constraints and challenges that they and their communities face, and how they might resolve them.

I examined documents and media (such as forms, photographs, labels, databases, datasets and reports) to gain insight into institutional norms or expectations.
My analysis emphasized how people interacted with these objects, so that I could assess how they valued them and the conditions under which they deemed them useful or meaningful.
I also examined documents and media as means for encapsulating and communicating meanings among users across space and over time.
This helped me to understand the vectors through which participants either tacitly form collective experiences or directly collaborate among themselves \parencites[]{huvila2011}[]{huvila2016}[]{yarrow2008}.
Some of my primary foci are understanding how document design and media capture protocols anticipate certain methods; how various activities refer to recorded information, especially archived information; the reasons why team members ignore certain equipment and forms of documentation despite their availability; how record-keeping is controlled through explicit or implicit imposition of limitations or constraints; why certain records play more a more central role than others; and how different archaeologists record the same objects in different ways.

Finally, my field notes comprised reflexive journal entries that I wrote between observational sessions or interviews.
They also include moments from observational sessions or interviews that I deemed particularly important, as well as descriptive accounts of unrecorded activities or conversations that I have since deemed useful data in their own right.

I obtained informed consent from all individuals included in this study in compliance with the University of Toronto's Social Sciences, Humanities, and Education Research Ethics Board, Protocol 34526.
In order to ensure that participants could speak freely about their personal and professional relationships while minimizing risk to their personal and professional reputations, I committed to refrain from publishing any personally identifying information.
I refer to all participants, affiliated organizations, and mentioned individuals or organizations using pseudonyms.
I also edited visual media to obscure participants' faces and other information that might reveal their identities, and took care to edit or avoid using direct quotations that were cited in other published work that follows a more permissive protocol regarding the dissemination of participants' identifying information.

\subsection*{Analysis}
I analyzed recorded observations and interviews, and interrogated the roles and affordances of various tools and documents, using qualitative data analysis methods.
More specifically, draw from the ``constellation of methods'' that Charmaz \parencite*[14-15]{charmaz2014} associated with grounded theory, namely coding and memoing.
See Batist \parencite*[9-10]{batist2024a} for a more comprehensive overview of the analytical methods employed for the project from which this paper emerges.

I refer to specific observations or interview segments throughout the rest of this text using references that resemble sequential endnotes, which are indexed in the Supplementary Materials.

% \subsection{Approach}
% This study is informed by a set of theoretical and methodological frameworks formed within a more interdisciplinary ``science studies'' tradition, which contribute to a more sociological outlook on archaeology as cultural practice \parencite[cf.][]{pickering1992}.
% In practical terms, I documented the social and collaborative experiences involved in various research practices, which ultimately bind the many ways in which archaeologists do archaeology.
% I specifically focused on the information commons, comprising both formal documents and mutually-held situated experiences that contribute to the development of shared understanding among archaeologists.
% This involved closely examining the ways in which researchers rely on physical and conceptual apparatus to capture, store, maintain and transmit information about the objects that captivate their interests.

% Previous work examining the apparatus of archaeological knowledge production relies heavily on Actor-Network Theory (ANT) or similar approaches that highlight the agency and impacts of tools on research practice.
% These draw attention to how non-human objects not only frame how human beings inhabit the world, but ``push back'' upon human actions with significant effects \parencite[]{latour1992}.
% In the context of early studies of scientific research, ANT was used to understand the physical and communicative mechanisms -- made up of non-human agents and information objects -- upon which scientists rely to capture, document and ascribe meaning to particular facets of the world \parencite[cf.][]{latour1986}.
% ANT posits that scientists can only identify, characterize and understand objects of interest by co-creating their conceptions of reality alongside non-human agents.

% However, since its inception 45 years ago, ANT has inspired a myriad of additional approaches.
% I primarily draw from the work of \textcite[]{knorrcetina1999}, who offers a more humanistic perspective by highlights how every action that someone takes in the production of knowledge is underpinned by a desire to fill a gap in knowledge or to square away any irregularities that disrupt ordered accounts of the world.
% Knorr Cetina's work builds upon Latour and Woolgar's \parencite*[]{latour1986} ethnography of laboratory settings with interviews of lab technicians concerning their thought processes while they worked.
% Specifically, \textcite[]{knorrcetina2001} emphasizes how scientists alter the material assemblage of the system on the basis of their understanding of what has or has not worked before, their suppositions concerning ways various actors might interact, and reiteration of their goals.
% Knorr Cetina therefore prioritizes human agency over the agency of non-human entities, which Latour and Woolgar, on the other hand, place on equal footing.

% Knorr Cetina's work reveals how ANT does not adequately account for the circumstances through which the structures that support science come into being, nor the intentionality of human agents who assemble material apparatus to meet their goals \parencite[cf.][]{whittle2009}.
% She refocuses attention on discursive aspects of knowledge production by considering expressions of potentiality, certainty and desire elicited by scientists as subjects acting with intent.
% This perspective, which emphasizes the perspectival and pragmatic aspects of scientific processes, is closely aligned with the social interactionist sociological framework, which originates from the pragmatist school of philosophy initiated by Charles Sanders Peirce, John Dewey, and George Herbert Mead.
% Social interactionism posits that people continuously re-create meanings through their shared understanding and interpretations of things \parencite[]{nungesser2021}.
% It therefore traces the construction of symbolic worlds from everyday interactions.

% This approach also aligns with the situated cognition methodological framework for examining the improvised, contingent and embodied experiences of human activity \parencites[cf.][]{suchman2007}[]{haraway1988}.
% It prioritizes subjects' outlooks, which are contextualized by their prior experiences, and enables scholars to trace how people make sense of their environments and work with the physical and conceptual tools available to them to resolve immediate challenges.
% Situated cognition therefore lends itself to investigating rather fluid, open-ended and affect-oriented actions, and is geared towards understanding how actors draw from their prior experiences to navigate unique situations.
% It is especially salient in explorations of how people who are learning new skills learn how to work in new and possibly unfamiliar ways, and in this sense is closely related to Lave and Wenger's \parencite*[]{lave1991} theory of situated learning (or `communities of practice' approach), which focuses on how individuals acquire professional skills in relation to their social environments.
% In such situations, situated cognition enables observers to examine how people align their perspectives as work progresses, and to understand better how people's general outlooks may have changed under the guidance of more experienced mentors.
% In other words, situated cognition enables researchers of scientific practices to account for discursive aspects of work, including perceived relationships, distinctions or intersections between practices that professional or research communities deem acceptable and unacceptable, and the cultural or community-driven aspects of decisions that underlie particular actions.

% In taking on this theoretical framework, I frame archaeology as a collective endeavour to derive a coherent understanding of the past, which involves the use of already established knowledge in the validation of newly formed ideas, and which relies on systems designed to carry information obtained with different chains of inference.
% These systems have both technical and social elements.
% The technical elements are the means through which information becomes encoded onto information objects so that they may form the basis for further inference.
% The social elements constitute a series of norms or expectations that facilitate the delegation of roles and responsibilities among agents who contribute their time, effort and accumulated knowledge to communal goals.

% As such, in constructing the arguments of this study and in carrying out the fieldwork that grounds it, I rely upon both realist and constructivist viewpoints.
% In one sense, I rely on documenting how people actually act, including the longer-term and collaborative implications that their actions may have on other work occurring throughout the continuum of practice.
% To accomplish this, I identify research activities from the perspective of an outside observer.
% I also ascribe meanings to things (such as physical or conceptual tools, or objects that captivate subjects' interests) in ways that conform to my own perspective as an investigator of scientific research practices.

% On the other hand, a constructionist perspective enables me to consider how individual agents make components of information systems suit their needs to facilitate communication or interoperability among actors who hold different situated perspectives. By listening to participants' views about the systems with which they engage, including explanations as to why they act in the ways that they do, I am able to trace the assumptions and taken-for-granted behaviours that frame their perspectives. Moreover, these insights are useful for developing a better understanding of how participants identify with particular disciplinary communities and their perception of their roles within broader collective efforts.

% Ultimately, this study is about the social order of scientific research, i.e. the frameworks, mindsets or sets of values that humans adopt to carry out their work in specific ways.
% Human beings rely upon physical and conceptual apparatus to do this work but, in order to understand how they do science \textit{in ways that conform to the epistemic mandates of the scientific enterprise}, it is necessary to prioritize attention to human intention, drivers and pressures.
% I am emphasizing the agency of human drivers since they are the ones who (a) identify problems that need to be resolved; (b) imagine, project or predict potential outcomes of various kinds of actions that they may select to resolve the challenges; and (c) learn from prior experiences and change their behaviours accordingly.
% By highlighting how pragmatic actions are conducted in relation to broader discursive frameworks, I consider scholarly practices in terms of potential, certainty and desire from the perspectives of practitioners themselves.
% This is made possible by considering data as discursive media that connects distributed actions experienced by people operating in disparate work environments, as described in Section 2.2 above.