This work builds upon prior studies of archaeological documentation in fieldwork settings, particularly Edgeworth's \parencite*[28]{edgeworth1991} dissertation that documented ``the transaction between the subject and the object, as it takes place in the act of discovery,'' which represented an attempt to ground theoretical discourse concerning the objectivity of the archaeological record in the practical ``intersubjective work or labour upon material objects''.
This extremely polyvalent work touched on various aspects of archaeological practice, highlighting the collective and discursive process of archaeological knowledge production in various settings.
Edgeworth closely examined the physical acts of excavation, the mindsets of the people doing this work, the sensory and conceptual apparatus through which objects are uncovered and made meaningful, and the social transactions that surround and permeate life on the project.
His work drew attention to the social and professional interactions taking place at an archaeological excavation, and which occur as archaeologists articulate an object as a meaningful or discrete entity and make it official.
Crucially, Edgeworth highlighted how archaeological records are produced through improvised, semi-structured and discursive action, afforded by practical concern and limited by the prior experiences held by those doing the work.

Similarly, \textcites[]{goodwin1994}[]{goodwin2010} observed how the formation of concrete records in fieldwork settings relates to the establishment of professional frameworks, which lend authoritative legitimacy to the meanings that archaeologists eventually settled upon.
This touched on similar observations made by \textcite[]{gero1996}, who noted how certain ways of delimiting features -- which corresponded with gendered experiences -- were deemed more legitimate than others.
\textcites[]{mickel2021}[]{yarrow2008} also showed a strong relationship between the diminished interpretative agency among archaeological labourers (including local labourers and undergraduate students) and their inability to contribute tangible and meaningful documentary records about the things they recover.

\textcite[]{thorpe2012} also argued that the broader social and political circumstances -- neoliberal austerity, in particular -- in which archaeological fieldwork tends to operate significantly effects how interpretations are made and arguments are extended, by effectively curtailing fieldworkers' creative agency.
\textcites[]{huggett2022}[]{caraher2019}[]{batist2021}[]{batist-alienation} similarly draw attention to how digital workflows effectively segregate acts of recording from acts of analysis and interpretation, by putting significant epistemic distance between those who hold creative agency in analytical and interpretive domains and those who occupy the domain of fieldwork; they further demonstrate how the latter is leveraged by the former to produce a clear and concise basis upon which formal analytical methods rest.
Moreover, \textcites[]{batist2024a}[]{haciguzeller2021} point out that the formal and transactional paradigm that dominates discourse on what data are and how they should be handled poses problems for communicating what was actually encountered while excavating a feature, including tentative thoughts, desires and apprehensions that are left out of official records.

In what follows, I will extend this critique by showcasing the improvised nature of data construction in fieldwork settings and by demonstrating how rough encounters with archaeological remains are stablizied and made more legitimate through documentation practices.

% ---
% Most notably, Hodder's work promoting reflexivity was primarily concerned with closing the hermeneutic loop between observation and interpretation.
% By embedding analysts in fieldwork settings and taking seriously the trench-side discussions that ultimately develop into solid and meaningful claims about the things being uncovered, he and his team developed the notion that archaeology is a discursive practice whereby archaeologists play an active role in the formation of archaeological facts.
% For instance, he recognized that the sheer competitiveness of academic archaeology, as well as archaeology's deep colonial roots, have profound impacts on the production ane legitimization of knowledge.
% While Hodder recognized that these contemporary social factors contribute to the interpretive process, and certainly opened the door for more robust critical discourse on the nature of archaeological knowledge claims, it is unclear whether he was actually effective at demonstrating that his project acted any differently than the norm.
% For instance, he claimed that giving fieldworkers more information about other people's interpretations of an object of interests would encourage more open-ended interpretative potential, but fieldworkers were not necessarily empowered to act on that information.
% This is given away in Hodder's later work (Entangled), in which he traces the entangled conceptual relations that effectively anchor knowledge claims through some form of scholarly concensus (largely mirroring Wylie's notions of cabling and tacking), but neglects to adequately account for the social and political backing that underpin these conceptual relations.
% Ideas require people to carry them forward, but not everyone is granted equal opportunity to participate in discursive action.

% This has been noted by Yarrow and Mickel, who identified those whose potential to contribute to meaningful archaeological interpretation is undercut by their social standing.
% Yarrow described how field assistants are more confident in their knowledge claims once they are welcomed to see the inner workings of archaeological paperwork.
% By beginning to see archaeology as an interpretive act, they overcome the drone-like and programmatic behaviour that they initial feel resigned to.
% Similarly, Mickel highlighted how local fieldworkers became much more confident to speak out about their interpretations once they were given the camera to use as they wished, rather than simple to just follow instructions.
% These studies demonstrate the impact that autonomy has on the ability to speak up.


% However, archaeological recording is wound up in mechanisms of control.
% As Huggett 2022 (276-278) noted, archaeological recording sheets are part of an apparatus through which distance is imposed between the analyst and the object of interest, by means of abstraction and stabilization.
% This reflects similar mechanisms documented in ecology by Star and Griesemer (1989).
% Archaeological recording, by default, is therefore a vector of control, wielded by the project director, to stabilize the record, to make it all click, to track work, to ensure that work does not have to be re-done, to encourage efficiency, to normalize interpretations.
% So designing and using recording sheets is not just about recording something, but about affixing a stable character, making an interpretation official.
