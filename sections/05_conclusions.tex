My findings demonstrate how the production of stable and concrete archaeological records involves characterizing the phenomena of interest in nominal terms, while downplaying the situated and embodied experiences that informed the records' creation.
The paper shows how these values are instilled through the social and material experiences in which fieldwork is embedded, which inform students about how their labour, and the outcomes of their labour, contribute to collective efforts.

More specifically, I observed a tendency toward enforcing formally-defined records in support of analytical tasks down the line.
Fieldwork is therefore presented as a means to an end, and fieldworkers are accordingly rendered as instruments that can be wielded to support future analytic endeavours.
This reveals how the management of archaeological data and of archaeological labour are inherently intertwined.
Consequently, the mechanisms through which archaeological projects establish control over those whose labour produces data bear broader epistemic implications regarding the nature and use of evidence in our reasoning about the past.

To be clear, the instrumentalization of archaeological labour is not necessarily a bad thing.
Information commons, such as the pool of knowledge accumulated throughout an archaeological project, do not necessarily have to be egalitarian, and are always governed by norms and expectations concerning who may contribute to and extract from communal resources, and in what ways these interactions should occur.
The fieldworkers I spoke with (including those whose elicitations do not appear in this paper; see \cite{batist2023a}, which is the broader dissertation from which this paper derives) generally valued their contributions as sensory devices.
This is linked to the idea that fieldworkers are capable of seeing things as they really are -- as material entities that have seemingly not yet been ascribed stable meaning.
As such, fieldworkers actively contributed to honing the illusion of their objectivity, which enhanced their value as members of the project and as domain specialists with their own unique mental skills.
At the same time, it was also clear that fieldworkers knew, on an intuitive level, that any claim of objectivity is overstated \parencite[12]{batist2024a}.
However, their positions as responsive rather than creative actors \parencite[cf.][]{batist-alienation} ensured that they are not responsible for resolving this tension.

The tension between the desire to achieve a state of objective sensor, and the inherently situated nature of observing and recording things, persists.
All observation is embodied, and all records carry biases imposed by the practical circumstances of their creation.
It is unclear how, or even if, this can can be resolved -- but it may perhaps be eased by fostering a commensal and social attitude toward data-sharing, instead of the formal and transactional paradigm that underpins most open data infrastructures.


