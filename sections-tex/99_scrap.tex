
Even approaches that claim to close the loop on archaeological data curation (which purportedly helps with rates of reuse) reify a single-track and linear framework for data management, from earth to archive.

In reality, and as Alison Wylie demonstrated back in 1989, archaeological knowledge is produced through multilinear channels of reasoning, in what Dallas (2015) and Batist (2023) refer to as a continuum of practice woven together by pragmatic acts of communication.

While these connections are in part driven by localized and situated experience, the social, technological and institutional structures upon which 21st century archaeology is based also play major roles in shaping these connections.

In particular, there is a pressure to force the information we collect to fit into relatively rigid frameworks, to render data as confident representations of reality, as objective embodiments of natural truth.

This is accomplished by putting distance between the observer and the object of interest, using methodologies, protocols, workflows and standardized interfaces.

The goal of this approach is to minimize the agency of the people who directly engage with the objects of interest, and represent each observation as a discrete instance of an abstract model (see Batist alienation).
However, as any experienced fieldworker will tell you, things on the ground are more fuzzy then they appear on paper.

Nevertheless, this is considered a necessity of professional practice, of producing legitimate findings.
Here I interrogate the this tension between fuzzy reality and concrete abstractions by articulating a series of observed processes through which archaeological observations are rendered into discrete records.