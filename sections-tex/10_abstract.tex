Archaeology comprises both systematic and improvised attitudes and processes concerned with the collection and maintenance of data.
This reflects the need to obtain formally-defined data, while also grappling with the fuzzy and uncertain nature of archaeological encounters, especially in fieldwork environments.
This produces an epistemic tension, whereby archaeologists struggle to reconcile their desire to produce concrete outcomes based on objective facts, and their intuitive understanding that data are in fact products of situated decisions and actions.
Through observations of archaeological practices, interviews with archaeologists at work, and analysis of the documents they produced while recording objects of archaeological concern, this paper articulates how archaeologists cope with this tension and integrate it into their work experiences.

