% Options for packages loaded elsewhere
\PassOptionsToPackage{unicode}{hyperref}
\PassOptionsToPackage{hyphens}{url}
\documentclass[
]{article}
\usepackage{xcolor}
\usepackage{amsmath,amssymb}
\setcounter{secnumdepth}{-\maxdimen} % remove section numbering
\usepackage{iftex}
\ifPDFTeX
  \usepackage[T1]{fontenc}
  \usepackage[utf8]{inputenc}
  \usepackage{textcomp} % provide euro and other symbols
\else % if luatex or xetex
  \usepackage{unicode-math} % this also loads fontspec
  \defaultfontfeatures{Scale=MatchLowercase}
  \defaultfontfeatures[\rmfamily]{Ligatures=TeX,Scale=1}
\fi
\usepackage{lmodern}
\ifPDFTeX\else
  % xetex/luatex font selection
\fi
% Use upquote if available, for straight quotes in verbatim environments
\IfFileExists{upquote.sty}{\usepackage{upquote}}{}
\IfFileExists{microtype.sty}{% use microtype if available
  \usepackage[]{microtype}
  \UseMicrotypeSet[protrusion]{basicmath} % disable protrusion for tt fonts
}{}
\makeatletter
\@ifundefined{KOMAClassName}{% if non-KOMA class
  \IfFileExists{parskip.sty}{%
    \usepackage{parskip}
  }{% else
    \setlength{\parindent}{0pt}
    \setlength{\parskip}{6pt plus 2pt minus 1pt}}
}{% if KOMA class
  \KOMAoptions{parskip=half}}
\makeatother
\setlength{\emergencystretch}{3em} % prevent overfull lines
\providecommand{\tightlist}{%
  \setlength{\itemsep}{0pt}\setlength{\parskip}{0pt}}
\usepackage[]{biblatex}
\usepackage{bookmark}
\IfFileExists{xurl.sty}{\usepackage{xurl}}{} % add URL line breaks if available
\urlstyle{same}
\hypersetup{
  hidelinks,
  pdfcreator={LaTeX via pandoc}}

\author{}
\date{}

\begin{document}

\section{Background}\label{background}

This work builds upon prior studies of archaeological documentation in
fieldwork settings, particularly Edgeworth's
\autocite*[28]{edgeworth1991} dissertation that documented ``the
transaction between the subject and the object, as it takes place in the
act of discovery,'' which represented an attempt to ground theoretical
discourse concerning the objectivity of the archaeological record in the
practical ``intersubjective work or labour upon material objects''. This
extremely polyvalent work touched on various aspects of archaeological
practice, highlighting the collective and discursive process of
archaeological knowledge production in various settings. Edgeworth
closely examined the physical acts of excavation, the mindsets of the
people doing this work, the sensory and conceptual apparatus through
which objects are uncovered and made meaningful, and the social
transactions that surround and permeate life on the project. His work
drew attention to the social and professional interactions taking place
at an archaeological excavation, and which occur as archaeologists
articulate an object as a meaningful or discrete entity and make it
official. Crucially, Edgeworth highlighted how archaeological records
are produced through improvised, semi-structured and discursive action,
afforded by practical concern and limited by the prior experiences held
by those doing the work.

Similarly, Goodwin \autocite*{goodwin1994,goodwin2010} observed how the
formation of concrete records in fieldwork settings relates to the
establishment of professional frameworks, which lend authoritative
legitimacy to the meanings that archaeologists eventually settled upon.
This touched on similar observations made by \textcite{gero1996}, who
noted how certain ways of delimiting features --- which corresponded
with gendered experiences --- were deemed more legitimate than others.
\textcite{mickel2021} and \textcite{yarrow2008} also documented that
archaeological labourers (including local labourers and undergraduate
students) are less able to contribute as interpretive agents in the
production of lasting records about the things they recover.

\textcite{thorpe2012} also argued that the broader social and political
circumstances --- neoliberal austerity, in particular --- in which
archaeological fieldwork tends to operate significantly effects how
interpretations are made and arguments are extended, by effectively
curtailing fieldworkers' creative agency. \textcite{huggett2022},
\textcite{caraher2019}, \textcite{batist2021} and
\textcite{batist-alienation} similarly draw attention to how digital
workflows effectively segregate acts of recording from acts of analysis
and interpretation, by putting significant epistemic distance between
those who hold creative agency in analytical and interpretive domains
and those who occupy the domain of fieldwork; they further demonstrate
how the latter is leveraged by the former to produce a clear and concise
basis upon which formal analytical methods rest. Moreover,
\textcite{batist2024a} and \textcite{haciguzeller2021} point out that
the formal and transactional paradigm that dominates discourse on what
data are and how they should be handled poses problems for communicating
what was actually encountered while excavating a feature, including
tentative thoughts, desires and apprehensions that are left out of
official records.

In what follows, I will extend this critique by showcasing the
improvised nature of data construction in fieldwork settings and by
demonstrating how rough encounters with archaeological remains are
stablizied and made more legitimate through documentation practices.

\printbibliography

\end{document}
