% Options for packages loaded elsewhere
\PassOptionsToPackage{unicode}{hyperref}
\PassOptionsToPackage{hyphens}{url}
\documentclass[
]{article}
\usepackage{xcolor}
\usepackage{amsmath,amssymb}
\setcounter{secnumdepth}{-\maxdimen} % remove section numbering
\usepackage{iftex}
\ifPDFTeX
  \usepackage[T1]{fontenc}
  \usepackage[utf8]{inputenc}
  \usepackage{textcomp} % provide euro and other symbols
\else % if luatex or xetex
  \usepackage{unicode-math} % this also loads fontspec
  \defaultfontfeatures{Scale=MatchLowercase}
  \defaultfontfeatures[\rmfamily]{Ligatures=TeX,Scale=1}
\fi
\usepackage{lmodern}
\ifPDFTeX\else
  % xetex/luatex font selection
\fi
% Use upquote if available, for straight quotes in verbatim environments
\IfFileExists{upquote.sty}{\usepackage{upquote}}{}
\IfFileExists{microtype.sty}{% use microtype if available
  \usepackage[]{microtype}
  \UseMicrotypeSet[protrusion]{basicmath} % disable protrusion for tt fonts
}{}
\makeatletter
\@ifundefined{KOMAClassName}{% if non-KOMA class
  \IfFileExists{parskip.sty}{%
    \usepackage{parskip}
  }{% else
    \setlength{\parindent}{0pt}
    \setlength{\parskip}{6pt plus 2pt minus 1pt}}
}{% if KOMA class
  \KOMAoptions{parskip=half}}
\makeatother
\setlength{\emergencystretch}{3em} % prevent overfull lines
\providecommand{\tightlist}{%
  \setlength{\itemsep}{0pt}\setlength{\parskip}{0pt}}
\usepackage[]{biblatex}
\usepackage{bookmark}
\IfFileExists{xurl.sty}{\usepackage{xurl}}{} % add URL line breaks if available
\urlstyle{same}
\hypersetup{
  hidelinks,
  pdfcreator={LaTeX via pandoc}}

\author{}
\date{}

\begin{document}

\section{Methods and Materials}\label{methods-and-materials}

This paper draws from observations of and interviews with archaeologists
at work, as well as the documents that they produced. Specifically, I
articulate how archaeologists enacted various activities and how their
actions were situated as part of broader systems of knowledge
production. My involvement with this project constituted a longitudinal
investigation of archaeological practice that contributed to my doctoral
dissertation \autocite[see][]{batist2023a}.

\subsection{Case}\label{case}

This paper draws from observations of and interviews with members of an
archaeological project, focusing specifically on the pragmatic and
multifaceted ways in which participants engage with the project's
information system. This involved recording and interviewing
archaeologists as they worked during the summer field seasons from 2017
to 2019 and holding additional interviews between fieldwork sessions.
The project's director also provided access to all documents and records
for the purpose of this research.

The project upon which this case is based is a research project
comprising excavation of a prehistoric site in Southern Europe. It is
directed by a foreign professor affiliated with a North American
university, but who has extensive experience working in the region. The
director coordinates various specialists whom he recruited for their
expertise in the interpretation of finds, a number of trench supervisors
who lead excavation and data collection activities, and excavators who
operate under the guidance of their assigned trench supervisors.

It is a research project involving archaeologists with varying degrees
of experience and coming from diverse professional backgrounds,
including those with extensive experience in the commercial sector (see
the supplementary materials for brief summaries of individuals'
background). It is not a field school, but it does rely to a large
extent on labour provided by undergraduate and graduate students, and as
such engages in on-the-job training. The informal and situated learning
experiences I was able to observe represent instances of legitimate
peripheral participation, whereby newcomers are introduced to the norms
and expectations that govern the archaeological community of practice.
These scenarios provide especially clear opportunities to ascertain the
value ascribed to observed practices and information outcomes conveyed
by teachers and adopted by learners through active and productive
tutelage. There is a strong precedent for this approach to the study of
archaeological practice
\autocites[cf.][]{everill2007,goodwin1994,goodwin2010,morgan2018},
including in contexts of continual, life-long learning among experienced
archaeologists \autocites[cf.][]{edgeworth1991,gero1994,graham2019}.

This project served as one of three cases I investigated for my doctoral
dissertation, which documents how archaeological information systems
scaffold the collaborative and epistemic commitments that govern
professional research practices \autocite{batist2023}. The observations
and elicitations I present in this paper illustrate a specific
phenomenon with a more refined scope than what is accounted for in that
more comprehensive work and in other research outcomes deriving from it
\autocites[cf.][]{batist-alienation,batist2024a,batist2021}. As such,
this paper draws from a relatively small portion of the entire set of
observations and interviews, which largely pertain to fieldwork
recording practices, processing and analysis of finds, records
management, interdisciplinary collaboration, decisions regarding writing
and publication of findings, and discussions of how data and findings
are presented, evaluated and revised among broader research communities.

I actively contributed to the project for several years, primarily
serving as a database manager. This afforded me with greater awareness
of its organizational and institutional history, including a deeper
familiarity with all the people involved, and provided me with a
privileged outlook on how team members structure information, how they
typically use data, and what circumstantial events or motivating factors
frame such concerns. My continual and participatory engagement with this
project allowed me to develop an understanding of the intricate social
relations as they developed over time, and enabled me to examine certain
methods that are drawn out over the course of several field seasons.

I must emphasize that in case-study research, cases represent discrete
instances of a phenomenon relating to a researcher's interest
\autocite{ragin1992}. Cases are therefore not the subjects of inquiry,
but the vehicles through which phenomena of interest are manifested in
an observable way. I recognize that all archaeological projects are
informed by their own histories, memberships, sets of tools, methods,
and social or political circumstances, which inform distinct traditions
of practice, and that it is not possible to generalize across the whole
discipline through a single case study. In other words, my findings are
informed by the informants whose actions and attitudes I sought to
articulate, and by my own perspective as a scholar of the culture and
practice of archaeology and of the media and infrastructures that
support it. The implication is that commercial archaeology, which
comprises the vast majority of archaeological work in North America and
Europe, is out of the study's scope, owing to the fact that the case
represents a research project and that I have very limited experience
with and knowledge about commercial archaeology. However, see
\textcite{chadwick1998}, \textcite{thorpe2012} and \textcite{zorzin2015}
for similar research pertaining to commercial archaeology which produced
complementary findings as those presented here.

That being said, this single case study does articulate some significant
factors that contribute to decisions and behaviours that archaeologists
commonly make and enact, makes certain underappreciated social and
collaborative commitments that underlie common tools and practices more
visible, and draws attention to certain patterns of practice that relate
to contemporary discourse on the nature of archaeological data and
ongoing development of information infrastructures.

\subsection{Data}\label{data}

My dataset comprises recorded observations, embedded interviews,
retrospective interviews, archaeological documentation, and ethnographic
and reflexive fieldnotes.

Observational data comprised records of participants' behaviours as they
performed various archaeological activities and take the form of video,
audio and textual files. They enable me to document \emph{how} practices
are performed, in addition to the fact \emph{that} they are performed.
Moreover, observational data allow me to document what participants
actually do as opposed to what they think or say they do. For instance,
I situated activities in relation to broader systems, even when
participants are unaware that they are contributing to these systems, in
order to consider how activities occurring at various times or in
various contexts indirectly relate to, compare with or inform each
other. I observed roughly 66 hours of work from this case, which were
typically recorded using three different cameras --- including cameras
placed on participants' foreheads --- to obtain different perspectives
on the recorded activities. Some of the primary foci that guided my
observations were the processes that result in archaeological records;
people's use of information objects or interfaces, which sometimes
differ from expected behaviour established through their design; how
subjects implemented unconventional solutions or ``hacks'' to work
around problems; how the context of an activity affects its
implementation; and how local or idiosyncratic terms, concepts and
gestures become established in a research community.

Embedded interviews comprised conversational inquiries with participants
in the context of their work, and were meant to account for
participants' perspectives regarding how and why they act as they do,
given the immediate constraints of the situation at hand. Embedded
interviews provided insight into the practicalities of work in the
moment, from the perspective of practitioners themselves
\autocite{flick1997,flick2000,witzel2000}. They are also useful for
comparing participants' responses with observational records to
interrogate how and why participants' observed actions may differ from
the rationales elucidated from embedded interviews. It is difficult to
quantify how much data were collected based on embedded interviews since
they occurred while observing work they are therefore difficult to
distinguish from the remainder of the interaction. Embedded interviews
focused on how participants identify problems or challenges in their
work, and to determine ways to resolve them; how certain people gain
recognition as domain experts or authorities with specialized knowledge;
how specialists relate their contributions to the contributions of
others; and how specialists relate their situated perspectives to
centralized knowledge repositories.

Retrospective interviews comprised longer sem-structured interviews
outside of work settings with select participants to contextualize data
collected by other means and to determine participants' views on more
general or relatively unobservable aspects of archaeological research
(such as planning, publishing, collaboration, etc). Participants were
selected for interviews based on the potential to triangulate different
perspectives on objects, themes or situations whose significance was
emerging throughout the research \autocite{morse2019}. I conducted 13
interviews during this case; some included more than one participant,
and some participants sat for more than one interview. Retrospective
interviews helped me gain insight into how participants situate
themselves as members of and in relation to research communities, which
may be characterized by different regimes of value and by different
methodological protocols or argumentation strategies. They were meant to
highlight participants' perspectives on the value of various kinds of
research outputs, what they value in their work and the work of others,
the major constraints and challenges that they and their communities
face, and how they might resolve them.

I examined documents and media (such as forms, photographs, labels,
databases, datasets, reports, instructional media and field manuals) to
gain insight into institutional norms or expectations. See
\textcite{batist2024} and \textcite{batist-alienation} for more in-depth
analysis of how people interacted with and valued these documents. This
involved examining documents and media as means for encapsulating and
communicating meanings among users across space and over time. This
helped me to understand the vectors through which participants either
tacitly form collective experiences or directly collaborate among
themselves \autocite{huvila2011,huvila2016,yarrow2008}. Document
analysis emphasized understanding how document design and media capture
protocols anticipate certain methods; how various activities refer to
recorded information; the reasons why team members ignore certain
equipment and forms of documentation despite their availability; how
record-keeping is controlled through explicit or implicit imposition of
limitations or constraints; why certain records play more a more central
role than others; and how different archaeologists record the same
objects in different ways.

Finally, my field notes comprised reflexive journal entries that I wrote
between observational sessions or interviews. They also include moments
from observational sessions or interviews that I deemed particularly
important, as well as descriptive accounts of unrecorded activities or
conversations that I have since deemed useful data in their own right.

I obtained informed consent from all individuals included in this study
in compliance with the University of Toronto's Social Sciences,
Humanities, and Education Research Ethics Board, Protocol 34526. In
order to ensure that participants could speak freely about their
personal and professional relationships while minimizing risk to their
personal and professional reputations, I committed to refrain from
publishing any personally identifying information. I refer to all
participants, affiliated organizations, and mentioned individuals or
organizations using pseudonyms. I also edited visual media to obscure
participants' faces and other information that might reveal their
identities, and took care to edit or avoid using direct quotations that
were cited in other published work that follows a more permissive
protocol regarding the dissemination of participants' identifying
information.

\subsection{Analysis}\label{analysis}

I analyzed recorded observations and interviews, and interrogated the
roles and affordances of various tools and documents, using qualitative
data analysis methods. More specifically, I draw from the
``constellation of methods'' that \textcite[14-15]{charmaz2014}
associated with grounded theory, namely coding and memoing. Coding
involves defining what data are about in terms (or codes) that are
relevant to the theoretical frameworks that inform my research, and
identifying instances of these concepts (codings) as they appear
throughout the text \autocite[43]{charmaz2014}. Memoing entails more
open-ended exploration and reflection upon latent ideas in order to
crystallize them into new avenues to pursue, and constitutes a
relatively flexible way of engaging with data and serves as fertile
ground for honing new ideas \autocite[72]{charmaz2014}. By creating
memos that relate and elaborate series of encoded observations and that
situate observed experiences in relation to broader theoretical
frameworks, I was able to form more robust and thematic arguments about
the phenomena of interest while remaining firmly grounded in the
empirical data. See \textcite[9-10]{batist2024a} for a more
comprehensive overview of the analytical methods employed for the
project from which this paper emerges.

Codes were generated through iterative analysis of recorded
observations, transcribed interviews and scans of documets and field
notes. An initial ``open'' coding phase involved generating codes in a
rather open-ended manner to identify common themes and sensitizing
concepts {[}cf.~Saldana xxx{]}. Subsequent coding was scaffoled by a
provisional taxonomic code system, which was informed by a loose
conceptual model, which broadly encompases codes about specific
activities, participants' figurations, and theoretical concepts
concernong the collective constitution of scienctific knowledge
\autocite[see][Appendix B for an overview of the code
system]{batist2023}. The study is therefore aligned with Charmaz's
\autocite*{charmaz2000} constructivist approach to grounded theory, in
that it orients the work by the analyst's prior understandings, rather
than have the codes emerge organically.

I refer to specific observations or interview segments throughout the
rest of this text using endnotes, which are indexed in the supplementary
materials.

\printbibliography

\end{document}
