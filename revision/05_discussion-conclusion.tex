% Options for packages loaded elsewhere
\PassOptionsToPackage{unicode}{hyperref}
\PassOptionsToPackage{hyphens}{url}
\documentclass[
]{article}
\usepackage{xcolor}
\usepackage{amsmath,amssymb}
\setcounter{secnumdepth}{-\maxdimen} % remove section numbering
\usepackage{iftex}
\ifPDFTeX
  \usepackage[T1]{fontenc}
  \usepackage[utf8]{inputenc}
  \usepackage{textcomp} % provide euro and other symbols
\else % if luatex or xetex
  \usepackage{unicode-math} % this also loads fontspec
  \defaultfontfeatures{Scale=MatchLowercase}
  \defaultfontfeatures[\rmfamily]{Ligatures=TeX,Scale=1}
\fi
\usepackage{lmodern}
\ifPDFTeX\else
  % xetex/luatex font selection
\fi
% Use upquote if available, for straight quotes in verbatim environments
\IfFileExists{upquote.sty}{\usepackage{upquote}}{}
\IfFileExists{microtype.sty}{% use microtype if available
  \usepackage[]{microtype}
  \UseMicrotypeSet[protrusion]{basicmath} % disable protrusion for tt fonts
}{}
\makeatletter
\@ifundefined{KOMAClassName}{% if non-KOMA class
  \IfFileExists{parskip.sty}{%
    \usepackage{parskip}
  }{% else
    \setlength{\parindent}{0pt}
    \setlength{\parskip}{6pt plus 2pt minus 1pt}}
}{% if KOMA class
  \KOMAoptions{parskip=half}}
\makeatother
\setlength{\emergencystretch}{3em} % prevent overfull lines
\providecommand{\tightlist}{%
  \setlength{\itemsep}{0pt}\setlength{\parskip}{0pt}}
\usepackage[]{biblatex}
\usepackage{bookmark}
\IfFileExists{xurl.sty}{\usepackage{xurl}}{} % add URL line breaks if available
\urlstyle{same}
\hypersetup{
  hidelinks,
  pdfcreator={LaTeX via pandoc}}

\author{}
\date{}

\begin{document}

\section{Discussion and Conslusion}\label{discussion-and-conslusion}

This paper's findings demonstrate how the production of stable and
concrete archaeological records involves characterizing the phenomena of
interest in nominal terms, while downplaying the situated and embodied
experiences that informed the records' creation. More specifically, it
documents a tendency toward enforcing formally-defined records in
support of analytical tasks down the line, which present fieldwork as a
means to an end, and fieldworkers as instruments that can be wielded to
support future analytic endeavours. It shows how these values are
instilled through the social and material experiences in which fieldwork
is embedded, which inform fieldworkers about how their labour, and the
outcomes of their labour, contribute to collective efforts. In other
words, it reveals how the management of archaeological data and of
archaeological labour are inherently intertwined, and draws attention to
some mechanisms through which certain voices are rendered more visible
than others when constituting the archaeological record.

Moreover, the fieldworkers I observed and spoke with played into the
roles they were assigned, even though this meant having less creative
agency. In fact, they generally valued their contributions as sensory
devices, which is linked to the notion that they were capable of seeing
things as they really are --- as material entities that have seemingly
not yet been ascribed stable meaning. As such, fieldworkers actively
contributed to honing the illusion of their objectivity, which enhanced
their value as members of the project and as domain specialists with
their own unique mental skills. At the same time, it was also clear that
fieldworkers knew, on an intuitive level, that any claim of objectivity
is overstated \autocite[as per complementary work published
in][12]{batist2024a}. However, their positions as responsive rather than
creative actors ensured that they are not responsible for resolving this
tension \autocite[cf.][]{batist-alienation}.

These findings complement other empiritical research examining
archaeological data management as collective interpretive action. For
instance, \textcite{batist2021} and \textcite{batist-alienation} note
that rote fieldwork practices tend to be assigned to relatively junior
project personnel, who become ensnared in workflows which discipline
their actions. Similarly, \textcite{morgan2018}, who compare analog and
digital field drawing techniques, reveal how the act of transcribing
archaeological deposits on a blank surface produces greater
understanding in the minds of students than participating as a cog in a
broader digital apparatus, owing to different degrees of creative
interpretive agency that each method affords them. Moreover,
\textcite{yarrow2008}, who examines the meaning, materiality and agency
in archaeological recording practices, draws attention to expressions of
resignation among fieldworkers who were more aware that the information
they record does not hold special meaning, based on their prior
undestanding of how these records would actually be used down the line.
\textcite{thorpe2012}, \textcite{zorzin2010}, \textcite{edgeworth1991}
and \textcite{watson2019} highlight similar perspectives in their
critical investigations of agentic relationships and knowledge
production in commercial archaeology.

This paper has clear implications for thinking about data documentation
and the potential for data to be re-used in secondary research contexts,
at distance from their contexts of creation. Surveys by
\textcite[299-301]{faniel2012}, \textcite[676-677]{atici2013},
\textcite[90-91]{kansa2013} and \textcite[213]{chapman2016}, which
investigated the needs of data re-users, highlighted their desires to
communicate directly with datasets' creators to ascertain the subtext
hidden between the lines of their formal documentation. This aligns with
investigations of attempts to enhance archaeological documentation by
\textcite{huvila2022b}, \textcite{austin2024} and \textcite{opitz2021},
which suggest that such efforts should be directed by specific contexts
of re-use. A common thread across these investigations on either end of
the archive is that effective data-sharing must involve some discursive
relationship between those who produce and re-use data, thereby bridging
the epistemic distance imposed by layers of abstraction
\autocite{huggett2022a}. Strategies for enhancing data's re-use
potential across the continuum of archaeological practice thereby embody
Dallas' \autocite*{dallas2015} notion of curation as simultaneous acts
of reconciliation and anticipation, whereby meanings are negotiated in
relation to prior and future objectives and circumstances.

However, the systemic drive to produce certain kinds of information
outcomes based on confident and stable data sources is not fully
compatible with this need to acknowledge the complex and storied
histories of data. This tension between distinct notions of data, as
concrete and disembodied records in one sense, and as situated products
of decisions, actions and circumstances in another, produces an
epistemic anxiety that archaeologists must cope with. It is unclear how,
or even if, this epistemic tension can be resolved, but the drive to
achieve a state of objectivity in fieldwork, which is facilitated by
systemic distributions of agency, persists --- in spite of this
outcome's impossibility --- as one coping mechanism.

To be clear, this is not necessarily a bad thing; the
instrumentalization of archaeological labour is often necessary in order
to derive concrete and confident records that are suitable for
analytical methods which comply with modern scientific quality
standards. Moreover, information commons, such as the pool of knowledge
accumulated throughout an archaeological project, do not necessarily
have to be egalitarian, and are always governed by norms and
expectations concerning who may contribute to and extract from communal
resources, and in what ways these interactions should occur. But rather
than lean in to the illusion of archaeological objectivity, which is a
value built in to most contemporary data management systems, it may be
prudent to try an alternative approach that fosters a commensal attitude
toward data; namely, one which more fully recognizes data work occurring
throughout the continuum of archaeological practice, including in
domains that are not typically recognized for their capacity to work
with data, such as fieldwork.

\printbibliography

\end{document}
