% Options for packages loaded elsewhere
\PassOptionsToPackage{unicode}{hyperref}
\PassOptionsToPackage{hyphens}{url}
\documentclass[
]{article}
\usepackage{xcolor}
\usepackage{amsmath,amssymb}
\setcounter{secnumdepth}{-\maxdimen} % remove section numbering
\usepackage{iftex}
\ifPDFTeX
  \usepackage[T1]{fontenc}
  \usepackage[utf8]{inputenc}
  \usepackage{textcomp} % provide euro and other symbols
\else % if luatex or xetex
  \usepackage{unicode-math} % this also loads fontspec
  \defaultfontfeatures{Scale=MatchLowercase}
  \defaultfontfeatures[\rmfamily]{Ligatures=TeX,Scale=1}
\fi
\usepackage{lmodern}
\ifPDFTeX\else
  % xetex/luatex font selection
\fi
% Use upquote if available, for straight quotes in verbatim environments
\IfFileExists{upquote.sty}{\usepackage{upquote}}{}
\IfFileExists{microtype.sty}{% use microtype if available
  \usepackage[]{microtype}
  \UseMicrotypeSet[protrusion]{basicmath} % disable protrusion for tt fonts
}{}
\makeatletter
\@ifundefined{KOMAClassName}{% if non-KOMA class
  \IfFileExists{parskip.sty}{%
    \usepackage{parskip}
  }{% else
    \setlength{\parindent}{0pt}
    \setlength{\parskip}{6pt plus 2pt minus 1pt}}
}{% if KOMA class
  \KOMAoptions{parskip=half}}
\makeatother
\setlength{\emergencystretch}{3em} % prevent overfull lines
\providecommand{\tightlist}{%
  \setlength{\itemsep}{0pt}\setlength{\parskip}{0pt}}
\usepackage[]{biblatex}
\usepackage{bookmark}
\IfFileExists{xurl.sty}{\usepackage{xurl}}{} % add URL line breaks if available
\urlstyle{same}
\hypersetup{
  hidelinks,
  pdfcreator={LaTeX via pandoc}}

\author{}
\date{}

\begin{document}

\section{Back Matter}\label{back-matter}

\subsection{acknowledgments}\label{acknowledgments}

I extend warm thanks to Costis Dallas, Matt Ratto, Ted Banning, Jeremy
Huggett and Ed Swenson for supervising my work and providing critical
feedback as I conducted my doctoral research, which this paper is based
upon. I am also very grateful to the anonymous reviewers for their
constructive evaluation. Of course, this work would not have been
possible without the anonymous research participants who allowed me to
observe and interview them as they worked, and to articulate their
actions and outlooks.

\subsection{Funding}\label{funding}

This work is derived from the author's doctoral dissertation,
\emph{Archaeological data work as continuous and collaborative practice}
\autocite{batist2023a}, which was supported by the Canadian Social
Sciences and Humanities Research Council Doctoral Fellowship (Award ID:
752-2019-2233).

\subsection{Competing Interests}\label{competing-interests}

The author states no conflicts of interest.

\subsection{Author Contribution}\label{author-contribution}

Zachary Batist is the sole author of this work. He defined the scope of
the study and identified suitable cases for inclusion, collected and
processed all data, performed analysis, interpreted the findings,
created all the figures, and wrote the paper.

\subsection{Informed Consent}\label{informed-consent}

Informed consent has been obtained from all individuals included in this
study, in compliance with the University of Toronto's Social Sciences,
Humanities, and Education Research Ethics Board, Protocol 34526.

\subsection{Data Availability
Statement}\label{data-availability-statement}

The data generated and analyzed during the current study are included in
this published article's supplementary files.

\subsection{Author Bio}\label{author-bio}

Zachary Batist obtained his PhD from the University of Toronto's Faculty
of Information. His research explores the collaborative commitments
inherent throughout archaeological practice, especially relating to data
management and the constitution of information commons. He currently
works as a Postdoctoral Researcher at the Department of Epidemiology,
Biostatistics and Occupational Health in the School of Public and Global
Health at McGill University, where he investigates the collaborative,
technical and administrative structures that scaffold data harmonization
in epidemiological research.

\subsection{Figure Captions}\label{figure-captions}

\begin{itemize}
\tightlist
\item
  Figure 1: Explanation of a potential context change using gestures and
  speech.
\item
  Figure 2: Discussion of a potential context change using gestures and
  speech.
\item
  Figure 3: Jane describes how she learned to recognize differences in
  the soil.
\item
  Figure 4: Transcribed section of a recording sheet describing the
  context addressed in the observed episode.
\item
  Figure 5: Sketch of the base of a trench, portraying the context
  addressed in the observed episode, boxed in red.
\item
  Figure 6: Section of a trench report describing the context addressed
  in the observed episode, and situating it as part of a
  lithostratigraphic unit.
\end{itemize}

\printbibliography

\end{document}
