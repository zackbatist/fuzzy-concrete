% Options for packages loaded elsewhere
\PassOptionsToPackage{unicode}{hyperref}
\PassOptionsToPackage{hyphens}{url}
\documentclass[
]{article}
\usepackage{xcolor}
\usepackage{amsmath,amssymb}
\setcounter{secnumdepth}{-\maxdimen} % remove section numbering
\usepackage{iftex}
\ifPDFTeX
  \usepackage[T1]{fontenc}
  \usepackage[utf8]{inputenc}
  \usepackage{textcomp} % provide euro and other symbols
\else % if luatex or xetex
  \usepackage{unicode-math} % this also loads fontspec
  \defaultfontfeatures{Scale=MatchLowercase}
  \defaultfontfeatures[\rmfamily]{Ligatures=TeX,Scale=1}
\fi
\usepackage{lmodern}
\ifPDFTeX\else
  % xetex/luatex font selection
\fi
% Use upquote if available, for straight quotes in verbatim environments
\IfFileExists{upquote.sty}{\usepackage{upquote}}{}
\IfFileExists{microtype.sty}{% use microtype if available
  \usepackage[]{microtype}
  \UseMicrotypeSet[protrusion]{basicmath} % disable protrusion for tt fonts
}{}
\makeatletter
\@ifundefined{KOMAClassName}{% if non-KOMA class
  \IfFileExists{parskip.sty}{%
    \usepackage{parskip}
  }{% else
    \setlength{\parindent}{0pt}
    \setlength{\parskip}{6pt plus 2pt minus 1pt}}
}{% if KOMA class
  \KOMAoptions{parskip=half}}
\makeatother
\setlength{\emergencystretch}{3em} % prevent overfull lines
\providecommand{\tightlist}{%
  \setlength{\itemsep}{0pt}\setlength{\parskip}{0pt}}
\usepackage[]{biblatex}
\usepackage{bookmark}
\IfFileExists{xurl.sty}{\usepackage{xurl}}{} % add URL line breaks if available
\urlstyle{same}
\hypersetup{
  hidelinks,
  pdfcreator={LaTeX via pandoc}}

\author{}
\date{}

\begin{document}

\section{Introduction}\label{introduction}

The series of challenges pertaining to the organization, sharing and
reuse of archaeological data, which are often collectively referred to
as the discipline's ``curation crisis'' or ``data deluge'', have
highlighted the wide array of practices that underlie data's
construction, management, dissemination and reuse
\autocite{bevan2012a,huggett2022,huggett2022a}. Numerous studies have
complicated the common imagination of data -- which considers them as
concise, corpuscular, discrete and inherently truthful records -- by
demonstrating how, in practice, they are actually messy, incomplete and
non-reductive
\autocites[cf.][]{batist2024a,dallas2015,huggett2022a,voss2012}. In
fact, archaeologists create data while anticipating their utility as
records that inform certain kinds of analysis, while those who apply
data in analytical contexts simultaneously reconcile their own use-cases
with the conditions under which the data were originally created
\autocite[190-191]{dallas2015}.

This is reflected in the ways in which archaeologists reuse data.
\textcite{faniel2013} and \textcite{atici2013} documented how those who
reuse data seek out additional contextual information about the
circumstances of a dataset's creation by communicating directly with the
dataset's originators, thereby establishing a discursive collaborative
tie. Alternatively, many data analysts who operate at a distance from
the contexts in which data originate prefer to trust in the models that
give the data concrete structure, thereby offloading the acts of
reconciliation to those who produced the data \autocite{huggett2022}. In
other words, reusing data involves establishing trust, which can be
garnered through mutual understanding of the challenges that had to be
overcome to get observations to fit within discrete data structures, or
through reliance on mechanisms of control to ensure that data are
collected and maintained in a consistent manner.

This paper demonstrates some of the strategies employed to establish
trust in data. Through analysis of one illustrative example of
archaeological documentation in fieldwork, I show how data-capture is
not merely a sensory experience whereby nature is recorded on a 1:1
basis, but is in fact structured by models and power relations that
legitimize data and make them useful.

\printbibliography

\end{document}
